% LaTeX source for the diSimplexEngine document
%

\documentclass[a4paper,openany]{amsbook}
\usepackage{perceptisys}
\usepackage{perceptisysChapter}
\usepackage{perceptisysSymbols}

\begin{document}
\frontmatter
\sloppy

\title[DiSimplicial Proof Theory]{Testing Mathematical Proofs: The details of
Directed Simplicial Proof Theory}
\collectionTitle{A Mathematical Theory of Reality}
\input{frontMatter}
\subjclass[2010]{Primary unknown; Secondary unknown} %
\keywords{Keyword one, keyword two etc.}%

\begin{abstract}
This is the abstract.
\end{abstract}

\maketitle
\tableofcontents
\mainmatter

\chapter*{Preface}

\begin{quotation}
We shall not cease from exploration \\
And the end of all our exploring \\
Will be to arrive where we started \\
And know the place for the first time.

\href{http://www.davidgorman.com/4Quartets/4-gidding.htm}{T. S. Eliot,
\textit{Little Gidding}, section `V'}
\end{quotation}

\vspace{0.5cm}

\begin{quotation}
Alice laughed. ``There's no use trying,'' she said: ``one can't believe
impossible things.'' \\ 
``I daresay you haven't had much practice,'' said the Queen. \\
``When I was your age, I always did it for half-an-hour a day. Why, sometimes
I've believed as many as six impossible things before breakfast.''

\begin{quote}
\href{http://en.wikiquote.org/wiki/Through_the_Looking-Glass#Chapter_5:_Wool_and_Water}{Lewis
Carroll (Charles Lutwidge Dodgeson), \textit{Through the Looking-Glass, and
What Alice Found There}, Chapter 5: \textit{Wool and Water}}
\end{quote}
\end{quotation}

\vspace{0.5cm}

\begin{quotation}
When you are a Bear of Very Little Brain, and you Think of Things, you find
sometimes that a Thing which seemed very Thingish inside you is quite different
when it gets out into the open and has other people looking at it.

\begin{quote}
\href{https://en.wikiquote.org/wiki/A._A._Milne#The_House_at_Pooh_Corner_.281928.29}{A. A. Milne,
\textit{House at Pooh Corner}, Chapter 6: \textit{In which Pooh invents a new
game and Eeyore joins in}}
\end{quote}
\end{quotation}

\vspace{0.5cm}

I \emph{am} a Bear of Very Little Brain, unless things are spelt out in great
detail, I am never quite sure if they are `correct'.  Being mildly dyslexic,
unless I can get a computer to keep track of these details, I am never sure if
\emph{I} might have got the details mixed-up.

This book is an attempt to spell out all of the details in sufficient detail
required to get computers to `check' \emph{informal} mathematical proofs as
typically written in standard research mathematical papers. The companion
papers, \cite{diSimplexTheory} and \cite{usingDiSimplexTheory}, provide a good
overview of the required theory and process respectively. It is highly likely
that the average reader will want to check the companion papers \emph{before}
diving into the details contained in this book\footnote{ Assuming the reader even
feels compelled to read this book at all}.

Most mathematicians regard providing every last detail required for a computer
to check a proof, as possible in principle, but essentially impossible in
practice. As an avid self-publicist, in that I do not intend to `publish' any of
my papers in a `traditional' `peer-reviewed' `journal', I need to be doubly sure
that my `proofs' are correct and not simply `wishful thinking'.

This cycle of work, of which this book and its companion papers is the
foundation, is intended to provide a \emph{Mathematical} theory of
\emph{Reality}. Essentially, this mathematical theory of reality, is an
exploration of the mathematics which a finite entity can compute in a finite
amount of time. As such, it is doubly important that the correctness of this
theory should be, both, \emph{checkable} and, moreover, \emph{be checked}, by a
finite entity such as a computer.

While I have, for many years, been planning on computationally checking all of
my proofs, it wasn't until discovering the recent work on Homotopy Type Theory
\cite{HoTT}, that I realized just how possible it might be. Unfortunately, as
will become obvious from this cycle of work, neither Per Martin-L\"of's
Dependent Type Theory, nor \emph{Homotopic} Type Theory are sufficient for my
purposes. Hence the need for this book to refound mathematics on \emph{Directed
Simplicial} Type Theory.  When the `dust settles' it will be obvious that both
\emph{Dependent} Type Theory and \emph{Homotopic} Type Theory will be
sub-theories of \emph{Directed Simplicial} Type Theory.

One of the key concepts of a mathematical theory of Reality, is that it is
\emph{critically} important to keep track of the informational complexity of a
mathematical object. The realization is that finite entities, such as ourselves,
can only \TODO{find correct word} with locally finite (mathematical) objects or
proceedures. This is the essential source of non-action at a distance in
Physics.

Classical mathematics, as well as most mathematical proof tools, ignore this
informational complexity. Indeed a real number has infinite complexity. 
Classical mathematical analysis, tends to allow itself to rely on woefully
non-locally finite tools. Directed Simplicial Type Theory and its associated
Proof engine are required to keep track of this informational complexity.

Members of the species \textit{Homeo Spaiens}, (as of this writting, ``you'' and
``me''), \emph{are} able to reason about mathematical objects of inifite
complexity.  Hence Directed Simplicial Type Theory \emph{must} be capable of
proving \emph{all} of mathematics. However, what is critical, is how the
constructive, locally-finite, mathematics is embedded in the classical,
non-locally-finite, mathematics. So we explicitly extend Per Martin-L\"of's
dependent type theory with impredicative, non-constructivist additions. As will
equally become clear in this work, Categorical thought, as opposed to classical
Zermelo–Fraenkel set theory, is the most `natural' way to found
mathematics\footnote{In my days in Nortel, the groups I worked with were
repeatedly asked why we did not use ``simple'' computer languages to takle our
Artificial Intelligent tasks. Our reply was always the same, while we
\emph{could} program in the language of a `raw' Turing machine, the complexity
of the code base would make our tools impossible to maintain. The point is that
different problems have `natural' languages in which humans can more easily
understand them. My argument for useing Directed Simplicial Category theory is
essentially the same. It is easier to think in a language which mimics
mathematic's focus on objects and transformations between objects than it is to
use ZFC. In fact, as we will see, Category theory is really topology \emph{as
well as} algebra, making the fundamentally topological Directed Simplicial
Category theory even more natural to an analysis of the space-time of
`Reality'.}.

\chapter{Introduction}

\section{Mathematics from a modipotent point of view}

Modipotent is `school girl' Latin derived from \emph{modi} meaning limited, and
\emph{potent} meaning power. Finite entities such as `you' and `me', are not
`gods'. We can only interact with `Reality' in locally-finite ways. While we are
not impotent, we have no omnipotent abilities to see all of reality or to
measure or sense things to inifinite precision. Some finite beings evidently
have more power then others, however we are all limited. Ignoring the nature of
these limitations in our mathematics is the source of the continuing paradoxical
nature of Quantum Mechanics as well as our current inability to unify
Realitivity with Quantum physics in a theory of Quantum Gravity.

Classical mathematics is essentially mathematics from an omnipotent point of
view. (Real) Magnitiudes can be measured to infinite precision. (Euclidian)
Points have no extension but, again, can be located in a Cartesian space to
inifinte precision.

Directed Simplicial Theory is an attempt to found mathematics in ways which
reflect our evident lack of omnipotence. 

\section{Type Theory}

We follow HoTT, \cite{HoTT}, by considering \emph{Types} as potentially having
more `structure' than a `simple' \emph{proposition}.  Instead of assuming a type
to be an abstract homotopic space, we define a \define{Directed Simplicial
Type}{Type} to be a Directed Simplicial Complex.

\section{Mathematical objectives}

One of our primary objectives as part of the mathematical theory of Reality, is
to be able to state and prove the imprecise Bayesian Theorem for
(predicative) Topos.  To do this we need to show the existence of imprecise
measures in (predicative) Topos. Any foundation of Topos theory requires among
other things the Yoneda lemma and Sheaf theory. So we start by sketching enough
Category theory to state and prove the Yoneda lemma.

However our first attainable goal is to prove

\begin{theorem}
The following conditions are equivalent:
\begin{itemize}
  \item $f$ is an isomorphism
  \item $f$ is epic and split monic
  \item $f$ is monic and split epic
\end{itemize}
\end{theorem}


\bibliographystyle{amsalpha}
\bibliography{diSimplexEngine}

\end{document}

